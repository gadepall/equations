\documentclass[12pt]{article}
\usepackage{amsmath}
\newcommand{\myvec}[1]{\ensuremath{\begin{pmatrix}#1\end{pmatrix}}}
\newcommand{\mydet}[1]{\ensuremath{\begin{vmatrix}#1\end{vmatrix}}}
\newcommand{\solution}{\noindent \textbf{Solution: }}
\providecommand{\brak}[1]{\ensuremath{\left(#1\right)}}
\providecommand{\norm}[1]{\left\lVert#1\right\rVert}
\let\vec\mathbf

\title{Qudratic Equation}
\author{Saipreet Pattjoshi (spattjoshi@sriprakashschools.com)}

\begin{document}
\maketitle
\section*{10$^{th}$ Maths - Chapter 4}
This is Problem-2 from Exercise 4.2
\begin{enumerate}
\item John and Jivanti together have 45 marbles. Both of them lost 5 marbles each, and the product of the number of marbles they now have is 124. We would like to find out how many marbles they have to start with.
\end{enumerate}
\solution \\
Given Data:\\
Let, the number of marbles John has = x.\\
Therefore, number of marbles Jivanti has = 45 – x\\
After losing 5 marbles each,\\
Number of marbles John has = x – 5\\
Number of marbles Jivanti has = 45 – x – 5 = 40 – x\\
Given,\\
 product of their marbles = 124.  Thus,
 \begin{align}
\brak{x – 5}\brak{40 – x} &= 124\\
\implies x^2-45x+324&=0
 \end{align}
 Thus,
\begin{align}
x &=\frac{-b\pm\sqrt{b^2-4ac}}{2a}\\
 &=\frac{-b\pm\sqrt{b^2-4ac}}{2a}\\
 &=\frac{45\pm\sqrt{-45^2-4 \times 1\times324}}{2 \times 1}\\
 &=\frac{45+\sqrt{2025-1296}}{2}\\
 &=\frac{45+\sqrt{729}}{2}\\
\end{align}
1st condition\\
\begin{align}
x &=\frac{45+27}{2}\\
 &=\frac{72}{2}\\
 &=36\\
\end{align}
2nd condition\\
\begin{align}
x &=\frac{45-27}{2}\\
 &=\frac{18}{2}\\
 &=9
\end{align}
Hence there roots are x=36 and x=9

\end{document}