\documentclass[12pt]{article}
\usepackage{amsmath}
\newcommand{\myvec}[1]{\ensuremath{\begin{pmatrix}#1\end{pmatrix}}}

\newcommand{\mydet}[1]{\ensuremath{\begin{vmatrix}#1\end{vmatrix}}}
\newcommand{\solution}{\noindent \textbf{Solution: }}
\providecommand{\brak}[1]{\ensuremath{\left(#1\right)}}
\providecommand{\norm}[1]{\left\lVert#1\right\rVert}
\let\vec\mathbf

\title{Linear Equations In Two Variables}
\author{K.Ruhaanth Karthikeya (koppuruhaanthkarthikeya@sriprakashschools.com)}

\begin{document}
\maketitle
\section*{Class No 9th} { Maths - Chapter 3}
This is Problem 1 from Exercise 3.1
\begin{enumerate}
\item Aftab tells his daughter, "Seven years ago,
I was seven times as old as you were then. Also, three years from 
now, I shall be three times as old as you will be." Represent this situation both algebraically and graphically.

\solution \\ Let us take Aftab's age as $x$ and his daughter's age as $y$
"Seven years ago I was seven times as old as you were.\\
Therefore,
\begin{align}
    x-7 = 7{\brak{y-7}}
\end{align}
so,
"Three years from now, I shall be three times as old as you will be."
Therefore,\\ 
\begin{align}
    x+3&= 3{\brak{y+3}}\\
x-7y &= -42. \\
x-3y&=6
\end{align}
The equations can also be written as:\\
\begin{align}
\myvec{1&-3\\1&-7}\myvec{x\\y}=\myvec{6\\-42}\\
\end{align}
\begin{align}
x =\frac{\mydet{ \vec{b} & \vec{a_2}}}{\mydet{ \vec{a_1} & \vec{a_2}}} = &
\frac{\mydet{6&-3\\-42&-7}}{ \mydet{1&-7\\1&-3}} =&
\frac{\brak{-6}\brak{-7} - \brak{-3}\brak{-42}}{\brak{1} \brak{-7}-
\brak{1}\brak{-3}} =& 42
\end{align}
Hence, Aftab's age is 42
\begin{align}
y = \frac{\mydet{ \vec{a_1} & \vec{b}}}{\mydet{ \vec{a_1} & \vec{a_2}}}=&
\frac{\mydet{1&6\\1&-42}}{ \mydet{1&-7\\1&3}}=&
\frac{\brak{1} \brak{-42} - \brak{1}\brak{6}}{\brak{1}\brak{-7} - \brak{1}\brak{-3}} =& 12
\end{align}
 Hence, Aftab's daughter's age is 12

\end{enumerate}

\end{document}

