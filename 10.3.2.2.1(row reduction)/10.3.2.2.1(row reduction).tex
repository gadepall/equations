\documentclass[10pt]{article}
\usepackage{amsmath}
\newcommand{\myvec}[1]{\ensuremath{\begin{pmatrix}#1\end{pmatrix}}}
\newcommand{\mydet}[1]{\ensuremath{\begin{vmatrix}#1\end{vmatrix}}}
\newcommand{\solution}{\noindent \textbf{Solution: }}
\providecommand{\brak}[1]{\ensuremath{\left(#1\right)}}
\providecommand{\norm}[1]{\left\lVert#1\right\rVert}
\let\vec\mathbf
\title{Linear Equations in Two Variables}
\author{Kishan (pusarlakishan@sriprakashschools.com)}
\begin{document}
\maketitle
\section*{10$^{th}$ Maths - Chapter 3}
This is Problem-4.1 from Exercise 3.2
\begin{enumerate}
\item On comparing the ratios $\frac{a_1}{a_2}$ , $\frac{b_1}{b_2}$ ,$\frac{c_1}{c_2}$, find out whether the lines representing the following pairs of linear equations intersect at a point, are parallel or coincident:\
$5x-4y+8=0$\\ 
$7x+6y-9=0$\\
\end{enumerate}
\solution\\
This can also be written as:
\begin{align}
\myvec{5&-4&-8\\7&6&9}
\end{align}
now,Making $R_2 \xrightarrow\ 5R_2 - 7R_1$\\
we get
\begin{align}
\myvec{5&-4&-8\\0&58&101}
\end{align}
now,making $R_1 \xrightarrow\ 2R_2 + 29R_1$\\
we get
\begin{align}
\myvec{145&0&-30\\0&58&101}
\end{align}
now,making $R_1 \xrightarrow\ R_1/(145)$\\
$R_2 \xrightarrow\ R_2/(58)$\\
we get
\begin{align}
\myvec{1&0&-30/145\\0&1&101/58}
\end{align}
so,$x=-30/145,y=101/58$\\
It is a independent equation.
\end{document}