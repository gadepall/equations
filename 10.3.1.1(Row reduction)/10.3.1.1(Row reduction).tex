\documentclass[12pt]{article}
\usepackage{amsmath}
\newcommand{\myvec}[1]{\ensuremath{\begin{pmatrix}#1\end{pmatrix}}}
\newcommand{\mydet}[1]{\ensuremath{\begin{vmatrix}#1\end{vmatrix}}}
\newcommand{\solution}{\noindent \textbf{Solution: }}
\providecommand{\brak}[1]{\ensuremath{\left(#1\right)}}
\providecommand{\norm}[1]{\left\lVert#1\right\rVert}
\let\vec\mathbf
\title{Linear Equations in Two Variables}
\author{ruhaanth(koppuruhaanthkarthikeya@sriprakashschools.com)}
\begin{document}
\maketitle
\section*{10$^{th}$ Maths - Chapter 3}
This is Problem-1 from Exercise 3.1
\begin{enumerate}
\item Aftab tells his daughter, "Seven years ago, I was seven times as old as you were then. Also, three years from 
now, I shall be three times as old as you will be." Represent this situation both algebraically and graphically.\\
\begin{align}
  x-7y&=-42\\
x-3y&=6\\
\end{align}

\end{enumerate}
\solution\\
This can also be written as:
\begin{align}
\myvec{1&-3&6\\1&-7&-42}
\end{align}
$R_2 \xrightarrow\ R_2 - R_1$\\ 
we get,
\begin{align}
\myvec{1&-3&6\\0&-4&-48}
\end{align}
$R_1 \xrightarrow\ 3R_2 - 4R_1$\\ 
we get,
\begin{align}
\myvec{-4&0&-168\\0&-4&-48}
\end{align}
$R_1 \xrightarrow\ R_1 $divided by -4\\
$R_2 \xrightarrow\ R_2 $divide\textsl{•}d by -4
we get,
\begin{align}
\myvec{1&0&42\\0&1&12}
\end{align}
Hence x=42 and y=12

\end{document}
